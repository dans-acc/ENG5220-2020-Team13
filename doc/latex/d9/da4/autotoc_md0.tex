\href{https://travis-ci.com/itsBelinda/ENG5220-2020-Team13}{\tt } \href{https://codecov.io/gh/itsBelinda/ENG5220-2020-Team13}{\tt } ~\newline
Currently in development.

The Bee\+Safe project is a low-\/cost, open source tracking device. It is designed to keep children safe by constantly monitoring their location, and alerting parents if the child leaves the designated safe zones. It consists of a Raspberry Pi 3B, and a purpose made P\+CB containing a u-\/blox S\+A\+R\+A-\/\+G350 G\+SM module, through which the Pi continuously gathers location data, and processes whether the child is in a location it is supposed to be at that time, and if the child is at a place they should not be, the device sends a text message to the parent with the child’s location they can track through a web interface.

To build our project, first clone the Bee\+Safe Github repository with ~\newline
 ` git clone \href{https://github.com/itsBelinda/ENG5220-2020-Team13}{\tt https\+://github.\+com/its\+Belinda/\+E\+N\+G5220-\/2020-\/\+Team13} ` ~\newline
 Next, run the installation script {\ttfamily install\+Bee\+Safe.\+sh} in the root folder of the repository\+: this will install all required dependencies. {\bfseries Note\+:} For optional dependencies, you can choose to install them or not by uncommenting or commenting them in the installation script. ~\newline
 Next, run the build script {\ttfamily build\+Bee\+Safe.\+sh} in the root folder of the repository\+: this will build the project and is ready to be ran. \begin{quote}
For more in-\/depth build instructions, please see the corresponding \href{https://github.com/itsBelinda/ENG5220-2020-Team13/wiki/Build-Guide}{\tt wiki page }. \end{quote}


To run the project as live project, ensure the build instructions were followed to create the bootloader service. If that was correctly implemented, the software should run upon the next startup. ~\newline
 To run the project in development, navigate into the folder {\ttfamily /software/\+Bee\+Safe\+P\+I/build/src/} in the repository, and type 
\begin{DoxyCode}
./BeeSafePI
\end{DoxyCode}
 \begin{quote}
For more in-\/depth usage instructions, please see the corresponding \href{https://github.com/itsBelinda/ENG5220-2020-Team13/wiki/User-Guide}{\tt wiki page }. \end{quote}


An in-\/depth guide on the build and usage of the system, as well as detailed component descriptions and project management steps can be found on the \href{https://github.com/itsBelinda/ENG5220-2020-Team13/wiki}{\tt project wiki}. The A\+PI documentation created with Doxygen can be found \href{}{\tt here}. \mbox{[}T\+O\+DO\mbox{]}

Do you like this project? We strongly believe in the power of Open Source, so there are many ways you could contribute to our project!

We are aware of the possibility that there may be bugs in the project, which were not immediately apparent for us (or \href{https://github.com/itsBelinda/ENG5220-2020-Team13/wiki/Project-Management%3A-Further-Work}{\tt were} \+:P ). If you find one, do let us know by opening an \href{https://github.com/itsBelinda/ENG5220-2020-Team13/issues}{\tt issue}, but please read our \href{https://github.com/itsBelinda/ENG5220-2020-Team13/blob/master/.github/ISSUE_TEMPLATE/bug_report.md}{\tt Bug Report Guidelines} first, to make it easier to work out the problem!

Is there a feature you would like to see implemented in this project? We welcome feature requests, however before you open one, please read our \href{https://github.com/itsBelinda/ENG5220-2020-Team13/blob/master/.github/ISSUE_TEMPLATE/feature_request.md}{\tt guidelines} for submitting one!

We welcome other contributions, too, so pull requests of patches, improvements, etc. are encouraged! By default we ask you pull the {\ttfamily master} branch. We only ask you to be mindful, and keep pull requests in scope and relevant. If you would be interested in contributing a larger piece of work and/or it may not be in line the above tips, please do get in touch with a \href{https://github.com/itsBelinda/ENG5220-2020-Team13/blob/master/README.md#credits}{\tt maintainer} for a discussion!

For the coding conventions we use, please look at the \href{https://github.com/itsBelinda/ENG5220-2020-Team13/wiki/Contributions#coding-conventions}{\tt wiki page} discussing it in depth!

If you have any, more general feedback about our project, give us a shout by filling out this \href{https://forms.gle/tGHM2jB7GBWfdgk3A}{\tt form}!

~\newline
 Feel free to also reach out to us via email, on beesafe.\+uofg \mbox{[}at\mbox{]} gmail.\+com!

Made with \+:heart\+: by {\bfseries Team Bee\+Safe}\+:~\newline
 \href{https://github.com/itsBelinda/}{\tt Belinda Kneubuhler}~\newline
 \href{https://github.com/szugyizs/}{\tt Zsuzsanna Szugyi}~\newline
 \href{https://github.com/dans-acc/}{\tt Daniels Vasiljevs}~\newline


\href{https://opensource.org/licenses/MIT}{\tt } ~\newline
 This project is licensed under the \href{https://github.com/itsBelinda/ENG5220-2020-Team13/blob/master/LICENSE}{\tt M\+IT license}. 