This page highlights the structure employed for defining virtual fences.

\section*{Fences}

Fences are defined according to the following hierarchical structure\+:



This was done due to each device account possessing numerous virtual fences. With respect to the implementations\+:


\begin{DoxyItemize}
\item \hyperlink{_fence_8h}{Fence.\+h} / .cpp defines common attributes i.\+e. whether the fence is regarded as being safe (the device can be removed), times, and the name.
\item \hyperlink{_round_fence_8h}{Round\+Fence.\+h} / .cpp defines a circular virtual fence around a centre point (latitude, longitude and radius). It does not however take into account the curvature of Earth.
\item \hyperlink{_poly_fence_8h}{Poly\+Fence.\+h} / .cpp defines polygon fences with numerous points. The implementation is based on the work published on \href{http://alienryderflex.com/polygon/?fbclid=IwAR1iLUkzQZnRRCWvPyUjvrYXU6W259FduDmq8NhDPSHMaUAtOmUnE_HEoAA}{\tt Determining Whether A Point Is Inside A Complex Polygon} -\/ which, once again, does not consider the curvature of the Earth.
\end{DoxyItemize}

Each instance is responsible for serialising J\+S\+ON elements that represent the fence (sub-\/classes), first invoke the super-\/class serialisation then build on this in the extended (overridden) functions. However, they are created using the \href{https://github.com/itsBelinda/ENG5220-2020-Team13/wiki/Account-Management}{\tt Account\+Builder}.

{\bfseries It should be noted that not taking the curvature into consideration only affects extremely large fences (areas larger than Texas).}

\subsection*{\hyperlink{_fence_8h}{Fence.\+h} / .cpp}

Not only does the \hyperlink{class_fence}{Fence} super-\/class define common attributes, but it also handles common operations such as determining whether the device needs to be present in the fence at a given time.


\begin{DoxyCode}
\textcolor{keywordtype}{bool} \hyperlink{class_fence_a7695b0f94f461369703188a287a38ab4}{Fence::isInTime}(\textcolor{keyword}{const} std::time\_t &time) \{

    \textcolor{comment}{// Extract information from system time.}
    std::tm time\_tm = *std::localtime(&time);
    \textcolor{keyword}{auto} iter = \hyperlink{class_fence_ae589e973fa03316847aeceedd72e2b64}{week}.find(time\_tm.tm\_wday);

    \textcolor{comment}{// Check if time information exists.}
    \textcolor{keywordflow}{if} (iter == \hyperlink{class_fence_ae589e973fa03316847aeceedd72e2b64}{week}.end()) \{
        \textcolor{keywordflow}{return} \textcolor{keyword}{true};
    \} \textcolor{keywordflow}{else} \textcolor{keywordflow}{if} (iter->second.empty()) \{
        \textcolor{keywordflow}{return} \textcolor{keyword}{true};
    \}

    \textcolor{comment}{// Iterate through days list of from and to times.}
    \textcolor{keyword}{auto} &dayTimes = iter->second;
    \textcolor{keywordflow}{for} (\textcolor{keyword}{const} std::pair<std::tm, std::tm> &dayTime : dayTimes) \{

        \textcolor{comment}{// If time is before from time, we are not present.}
        \textcolor{keywordflow}{if} (time\_tm.tm\_hour < dayTime.first.tm\_hour) \{
            \textcolor{keywordflow}{return} \textcolor{keyword}{false};
        \} \textcolor{keywordflow}{else} \textcolor{keywordflow}{if} (time\_tm.tm\_hour == dayTime.first.tm\_hour) \{
            \textcolor{keywordflow}{if} (time\_tm.tm\_min < dayTime.first.tm\_min) \{
                \textcolor{keywordflow}{return} \textcolor{keyword}{false};
            \}
        \}

        \textcolor{comment}{// If the time is after the to time, we are not present.}
        \textcolor{keywordflow}{if} (time\_tm.tm\_hour > dayTime.second.tm\_hour) \{
            \textcolor{keywordflow}{return} \textcolor{keyword}{false};
        \} \textcolor{keywordflow}{else} \textcolor{keywordflow}{if} (time\_tm.tm\_hour == dayTime.second.tm\_hour) \{
            \textcolor{keywordflow}{if} (time\_tm.tm\_min > dayTime.second.tm\_min) \{
                \textcolor{keywordflow}{return} \textcolor{keyword}{false};
            \}
        \}
    \}

    \textcolor{comment}{// By default the user is within the fence.}
    \textcolor{keywordflow}{return} \textcolor{keyword}{true};
\}
\end{DoxyCode}
 This is achieved by comparing the current time of the device (using {\ttfamily $\ast$std\+::localtime(\&time);}) against the serialised times represented by the {\ttfamily const std\+::map$<$int, std\+::vector$<$std\+::pair$<$std\+::tm, std\+::tm$>$$>$$>$ \&week}, where a day (the map key) maps onto a vector containing the starting and ending times ({\ttfamily std\+::pair$<$std\+::tm, std\+::tm$>$}). Moreover, in the event the said \hyperlink{class_fence}{Fence} map is empty, the function returns true (indicating that the device is indeed present in the fence at that particular time).

\subsection*{\hyperlink{_round_fence_8h}{Round\+Fence.\+h} / .cpp}

As stated, the round fence represents the fence defined around some coordinate (latitude, longitude). Thus, a determining whether the device is geographically located within can be achieved using simple geometry\+:


\begin{DoxyCode}
\textcolor{keywordtype}{bool} \hyperlink{class_round_fence_a4955bf0dbd168853ba9913465953865d}{RoundFence::isInLocation}(std::pair<double, double> &latLng)
\{
    \textcolor{keywordtype}{double} distance = std::sqrt((latLng.first - this->longitude) * (latLng.first - this->latitude)
                                + (latLng.second - this->longitude) * (latLng.second - this->longitude));
    \textcolor{keywordflow}{return} distance <= \hyperlink{class_round_fence_a8e9d1a2f22df0bb718522f3ab6cd3b83}{radius};
\}
\end{DoxyCode}
 Where the {\ttfamily std\+::pair$<$double, double$>$ \&lat\+Lng} represents the device coordinates ({\ttfamily lat\+Lng.\+first} and {\ttfamily lat\+Lng.\+second} represent the latitude and longitude, respectively); passed by the \hyperlink{_monitor_8h}{Monitor.\+h} / \hyperlink{_monitor_8cpp}{Monitor.\+cpp} (monitor thread).

\subsection*{\hyperlink{_poly_fence_8h}{Poly\+Fence.\+h} / .cpp}

Finally, polygon fences are represented using a vector of pairs i.\+e. {\ttfamily std\+::vector$<$std\+::pair$<$double, double$>$$>$ coordinates}; where each pair represents coordinates as follows\+:
\begin{DoxyItemize}
\item {\ttfamily lat\+Lng.\+first} and {\ttfamily lat\+Lng.\+second} represent the latitude and longitude, respectively.
\end{DoxyItemize}

\subsubsection*{Implementation}

With respect to the implementation of the algorithm as posited on \href{http://alienryderflex.com/polygon/?fbclid=IwAR1iLUkzQZnRRCWvPyUjvrYXU6W259FduDmq8NhDPSHMaUAtOmUnE_HEoAA}{\tt Determining Whether A Point Is Inside A Complex Polygon}, suggested optimisations have been taken into account. Thus, constants associated with the fence are calculated prior to monitoring being carried out i.\+e. during the initialisation phase of the device. The following code snippet posits the manner by which this is achieved\+:


\begin{DoxyCode}
\textcolor{keywordtype}{void} \hyperlink{class_poly_fence_a229de6f5987bf7d312310b522db0d5a4}{PolyFence::calculateFenceConstants}()
\{

    \textcolor{comment}{// Clear the constants for recalculation.}
    \hyperlink{class_poly_fence_a24c99bb8a45f86bdf51cd3f22ef0f174}{constants}.clear();
    \hyperlink{class_poly_fence_a2204e62b61b0e3c335734fa0b6cf0728}{multiples}.clear();

    \textcolor{comment}{// If the coordinates are empty, nothing can be done.}
    \textcolor{keywordflow}{if} (\hyperlink{class_poly_fence_ae8e0c55e745979cab104ef80aeb4b418}{coordinates}.empty()) \{
        \textcolor{keywordflow}{return};
    \}

    \textcolor{comment}{// Calculate poly fence constants.}
    \textcolor{keywordtype}{unsigned} \textcolor{keywordtype}{long} i, j = \hyperlink{class_poly_fence_ae8e0c55e745979cab104ef80aeb4b418}{coordinates}.size() - 1;
    \textcolor{keywordflow}{for} (i = 0; i < \hyperlink{class_poly_fence_ae8e0c55e745979cab104ef80aeb4b418}{coordinates}.size(); ++i) \{
        \textcolor{keywordflow}{if} (\hyperlink{class_poly_fence_ae8e0c55e745979cab104ef80aeb4b418}{coordinates}[i].second == \hyperlink{class_poly_fence_ae8e0c55e745979cab104ef80aeb4b418}{coordinates}[j].second) \{
            \hyperlink{class_poly_fence_a24c99bb8a45f86bdf51cd3f22ef0f174}{constants}.push\_back(\hyperlink{class_poly_fence_ae8e0c55e745979cab104ef80aeb4b418}{coordinates}[i].first);
            \hyperlink{class_poly_fence_a2204e62b61b0e3c335734fa0b6cf0728}{multiples}.push\_back(0);
        \} \textcolor{keywordflow}{else} \{
            \hyperlink{class_poly_fence_a24c99bb8a45f86bdf51cd3f22ef0f174}{constants}.push\_back(\hyperlink{class_poly_fence_ae8e0c55e745979cab104ef80aeb4b418}{coordinates}[i].first
                                - (\hyperlink{class_poly_fence_ae8e0c55e745979cab104ef80aeb4b418}{coordinates}[i].second * 
      \hyperlink{class_poly_fence_ae8e0c55e745979cab104ef80aeb4b418}{coordinates}[j].first)
                                  / (\hyperlink{class_poly_fence_ae8e0c55e745979cab104ef80aeb4b418}{coordinates}[j].second - 
      \hyperlink{class_poly_fence_ae8e0c55e745979cab104ef80aeb4b418}{coordinates}[i].second)
                                + (\hyperlink{class_poly_fence_ae8e0c55e745979cab104ef80aeb4b418}{coordinates}[i].second * 
      \hyperlink{class_poly_fence_ae8e0c55e745979cab104ef80aeb4b418}{coordinates}[i].first)
                                  / (\hyperlink{class_poly_fence_ae8e0c55e745979cab104ef80aeb4b418}{coordinates}[j].second - 
      \hyperlink{class_poly_fence_ae8e0c55e745979cab104ef80aeb4b418}{coordinates}[i].second));
            \hyperlink{class_poly_fence_a2204e62b61b0e3c335734fa0b6cf0728}{multiples}.push\_back((\hyperlink{class_poly_fence_ae8e0c55e745979cab104ef80aeb4b418}{coordinates}[j].first - 
      \hyperlink{class_poly_fence_ae8e0c55e745979cab104ef80aeb4b418}{coordinates}[i].first)
                                / (\hyperlink{class_poly_fence_ae8e0c55e745979cab104ef80aeb4b418}{coordinates}[j].second - 
      \hyperlink{class_poly_fence_ae8e0c55e745979cab104ef80aeb4b418}{coordinates}[i].second));
        \}
        j = i;
    \}
\}
\end{DoxyCode}
 Where determining whether or not the device is within the polygon fence is achieved by utilising the following function, analogous to the one posited by \hyperlink{_round_fence_8h}{Round\+Fence.\+h} / .cpp\+:


\begin{DoxyCode}
\textcolor{keywordtype}{bool} \hyperlink{class_poly_fence_af8116af5be86f8426102985c3dbcdf5e}{PolyFence::isInLocation}(std::pair<double, double> &latLng)
\{

    \textcolor{keywordtype}{bool} oddNodes = \textcolor{keyword}{false};
    \textcolor{keywordtype}{bool} current = \hyperlink{class_poly_fence_ae8e0c55e745979cab104ef80aeb4b418}{coordinates}.back().second > latLng.second;
    \textcolor{keywordtype}{bool} previous;

    \textcolor{comment}{// Determines whether or not latitude and longitude are within the fence.}
    \textcolor{keywordflow}{for} (\textcolor{keywordtype}{int} i = 0; i < \hyperlink{class_poly_fence_ae8e0c55e745979cab104ef80aeb4b418}{coordinates}.size(); ++i) \{
        previous = current;
        current = \hyperlink{class_poly_fence_ae8e0c55e745979cab104ef80aeb4b418}{coordinates}[i].second > latLng.second;
        \textcolor{keywordflow}{if} (current != previous) \{
            oddNodes ^= latLng.second * \hyperlink{class_poly_fence_a2204e62b61b0e3c335734fa0b6cf0728}{multiples}[i] + \hyperlink{class_poly_fence_a24c99bb8a45f86bdf51cd3f22ef0f174}{constants}[i] < latLng.first;
        \}
    \}

    \textcolor{keywordflow}{return} oddNodes;
\end{DoxyCode}
 Functionally, the algorithm compares each side of the polygon against the longitude of the device location. Additionally, a list of nodes are compiled, where each of the nodes represent a point where one side crosses the longitude threshold of the device location. Thus,


\begin{DoxyItemize}
\item If there are an odd number of points (on each side of the location), then device is said to be within the polygon.
\item If, however, there are an even number of nodes, the device location is outside of the polygon.
\end{DoxyItemize}

{\bfseries Reference\+:} \href{http://alienryderflex.com/polygon/?fbclid=IwAR1iLUkzQZnRRCWvPyUjvrYXU6W259FduDmq8NhDPSHMaUAtOmUnE_HEoAA}{\tt Determining Whether A Point Is Inside A Complex Polygon}

\subsection*{Displaying Virtual Fences}

{\itshape The following image depicts the defined virtual fences. It should be noted that due to time constraints, the web A\+PI backend was not implemented. Thus, the project falls just short of being completed due to the backend missing link for the website \mbox{[}frontend is present and is depicted by the following image\mbox{]}.}  