\subsection*{Continuous Integration}

To automatically check whether the project compiles correctly without having to carry out the C\+Make build generation step at every push to or pull from Git, the team decided to set up continuous integration (CI) for the project. This ensures correct operations and improves fault detection\+: by continuously and automatically testing every small change made on the program it is quicker and easier to discover faults, resulting in reduced flaw search and mend times, smaller feature backlog and better overall reliability. ~\newline
 The team chose Travis\+CI, a popular CI tool. It integrates with Github smoothly, and has a wide variety of settings available, allowing us to configure the run environment to match that of our specific Raspberry Pi. This ensures that if the project passes the Travis\+CI checks it will run properly on the actual hardware we use too, even if the hardware itself is not available for development temporarily. This also aids future contributions from users outside the maintaining team, as they can verify the correctness of their code without access to our specific device. It can be run for specific branches, and build options such as compilers, necessary packages, processing instructions can be set individually. ~\newline
 The project can be found here\+: \href{https://travis-ci.com/itsBelinda/ENG5220-2020-Team13}{\tt }

\subsubsection*{Setting up Travis\+CI}

To enable Travis\+CI to work with a project, the following steps need to be taken\+:
\begin{DoxyItemize}
\item Go to \href{https://travis-ci.com/signin}{\tt https\+://travis-\/ci.\+com/signin}
\item Choose \char`\"{}\+Sign in with Github\char`\"{}
\item Allow everything and enter your password until you\textquotesingle{}re redirected to your Github profile
\item There click \char`\"{}\+Activate\char`\"{} under the Github Apps Integration
\item There you\textquotesingle{}ll be presented with a list of repositories you can allow Travis\+CI to be installed for -\/ it\textquotesingle{}s set to all repos by default, so select \char`\"{}\+Only select repositories\char`\"{}, and select the project repository, then click approve and install
\item Once it\textquotesingle{}s done you\textquotesingle{}ll see a list of Github repos on your Travis\+CI profile
\item After this you only need to add a travis configuration file to the repository, and it is set.
\item To enable Travis\+CI to be run at every push and/or pull, click \char`\"{}\+More Options\char`\"{} and \char`\"{}\+Settings\char`\"{} and choose the appropriate settings
\end{DoxyItemize}

\subsubsection*{The Travis\+CI configuration file}

For Travis\+CI to detect a project (and branch) a file needs to be added to the root folder. The {\ttfamily .travis.\+yml} file needs to specify the language, the operating system, the operating system distribution, the compiler, any required packages that need to be installed, and any build commands. In case of using it with a code coverage tool, such as Codecov, the instructions to upload the code to Codecov’s servers need to be specified, too. ~\newline
 The {\ttfamily .travis.\+yml} file for this project is the following\+: 
\begin{DoxyCode}
language: cpp
dist: bionic
sudo: false

compiler:
# with CLang everything works fine, but gcov throws and out of memory error and doesn't upload to Codecov -
       might fix later if there's time
#  - clang
  - gcc

os: linux

addons:
  apt:
    packages: lcov

install:
  - sudo apt-get update
  - sudo apt-get install libcpprest-dev

before\_script:
  - cd software/BeeSafePI/build
  - cmake ..

script:
  - make
  - cd src/
  - ./BeeSafePI
  - cd ..
  - ctest

after\_success:
# Create lcov report
- lcov --capture --directory . --output-file coverage.info
- lcov --remove coverage.info '/usr/*' --output-file coverage.info # filter system-files
- lcov --list coverage.info # debug info
# Uploading report to CodeCov
- bash <(curl -s https://codecov.io/bash) -f coverage.info || echo "Codecov did not collect coverage
       reports"
\end{DoxyCode}
 In our case, it is important that the distribution is set to {\ttfamily bionic}, as otherwise the {\ttfamily libcpprest-\/dev} package is not detected. The {\ttfamily lcov} package also must be added as an addon, and the {\ttfamily libcpprest-\/dev} package must be installed before any of the build commands are run. ~\newline
 To match the build structure of the project, the cmake command must be called in the section {\ttfamily before\+\_\+script}, while the make and ctest commands can be called in {\ttfamily script}. ~\newline
 If using a code coverages service, such as Codecov, any and all commands related to that must be called in {\ttfamily after\+\_\+success}, as it requires a fully functional build before it can have a code coverage report created and uploaded to the service. ~\newline
 {\bfseries Note} \+: An error encountered in this project was that when using C\+Lang as the compiler, {\ttfamily gcov}, the parent package of {\ttfamily lcov}, a tool that helps create a code coverage report for the use of Codecov, throws an out of memory error, which causes the Travis build to run perfectly, however it does not upload to Codecov. When using gcc, the error is non-\/existent.

\subsection*{Unit testing}

Due to the sensitive nature of transmitting the child’s location live, and being able to promptly respond when the child leaves their intended safe zones, it is of key importance that the program running on the Pi is tested thoroughly for any flaws and has the maximum possible uptime. ~\newline
 As the project is an embedded project with limited memory available -\/ with even less if the planned future iterations are to go ahead with a Pi Zero -\/ , it was important to consider what non-\/necessary packages and software were downloaded. Due to this, the testing was decided to be done in a low-\/level, traditional manner, without any testing frameworks such as Boost or G\+Test. C\+Test is C\+Make’s own test tool, and it only requires for the checks of specific pass or fail values to be returned from a test, making it fast and reliable for an embedded project. By default these values are 0 for pass and 1 for fail, however this was changed to specific strings in this project for better readability. ~\newline
 First, the C\+Make configuration needed to be edited to allow for the inclusion of C\+Test, by adding 
\begin{DoxyCode}
enable\_testing()
add\_subdirectory(tests)
\end{DoxyCode}
 to the top level \hyperlink{_c_make_lists_8txt}{C\+Make\+Lists.\+txt} file, as it allows the detection of test files and they can then be built with the project. In the tests directory, which is on the same level as the source file directory, a \hyperlink{_c_make_lists_8txt}{C\+Make\+Lists.\+txt} file needed to be created to encapsulate the tests commands separately. As with unit tests code components need to be tested individually and to the lowest level, separate test files were created for each source class, making debugging easier. This, however resulted in over a dozen files that needed to be created. To avoid needing to edit the \hyperlink{_c_make_lists_8txt}{C\+Make\+Lists.\+txt} file with every file added or removed, a generic script was added to it that parses every file with a specific name (all test files follow the naming convention of “test\+\_\+” and name of the source class it tests), following this guide from Skand Hurkat\+: \href{https://skandhurkat.com/post/intro-to-ctest/}{\tt https\+://skandhurkat.\+com/post/intro-\/to-\/ctest/} . 
\begin{DoxyCode}
set(EXECUTABLE\_OUTPUT\_PATH $\{PROJECT\_BINARY\_DIR\}/tests)
set(CTEST\_BINARY\_DIRECTORY $\{PROJECT\_BINARY\_DIR\}/tests)

file(GLOB files "test\_*.cpp")

foreach(file $\{files\})
    string(REGEX REPLACE "(^.*/|\(\backslash\)\(\backslash\).[^.]*$)" "" file\_without\_ext $\{file\})
    add\_executable($\{file\_without\_ext\} $\{file\})
    target\_link\_libraries($\{file\_without\_ext\} $\{PROJECT\_LIBS\} PRIVATE cpprestsdk::cpprest)
    add\_test($\{file\_without\_ext\} $\{file\_without\_ext\})
    set\_tests\_properties($\{file\_without\_ext\}
        PROPERTIES
        PASS\_REGULAR\_EXPRESSION "Test passed")
    set\_tests\_properties($\{file\_without\_ext\}
        PROPERTIES
        FAIL\_REGULAR\_EXPRESSION "(Exception|Test failed)")
    set\_tests\_properties($\{file\_without\_ext\}
        PROPERTIES
        TIMEOUT 120)
endforeach()
\end{DoxyCode}


\subsection*{Testing improvements}

To verify the thoroughness of the tests, it is important to ensure as much of the software is covered by the tests as possible. A metric of this is using code coverage tools which run the tests and note what parts of the code are covered by checking what methods are called and what conditionals are chosen during the running of tests. Therefore, the project was released on one of the most popular code coverage services, which integrates with Github and Travis\+CI conveniently\+: Codecov.\+io. ~\newline
 The Codecov page for the project can be found here\+: \href{https://codecov.io/gh/itsBelinda/ENG5220-2020-Team13}{\tt } 